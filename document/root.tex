\documentclass[11pt,a4paper]{article}
\usepackage[T1]{fontenc}
\usepackage{isabelle,isabellesym}
\usepackage{amssymb}

% this should be the last package used
\usepackage{pdfsetup}

% urls in roman style, theory text in math-similar italics
\urlstyle{rm}
\isabellestyle{it}


\begin{document}

\title{Ramsey Number Bounds}
\author{Lawrence C. Paulson}
\maketitle

\begin{abstract}
Ramsey's theorem implies that for any given natural numbers $k$ and $l$, there exists some $R(k,l)$
such that a graph having at least $R(k,l)$ vertices must have either a clique of cardinality $k$
or an anticlique (independent set) of cardinality $l$. Equivalently, for a \emph{complete} graph of size $R(k,l)$,
every red/blue colouring of the edges must yield an entirely red $k$-clique or an entirely blue $l$-clique.
Although $R(k,l)$ is for practical purposes impossible to calculate given $k$ and $l$, 
some upper and lower bounds for it are known. The celebrated probabilistic argument by Paul Erdős is 
formalised here, with various of its consequences.
\end{abstract}

\newpage
\tableofcontents

\paragraph*{Acknowledgements}
Many thanks to Andrew Thomason and Chelsea Edmonds for their help with the probabilistic proofs.
The author was supported by the ERC Advanced Grant ALEXANDRIA (Project 742178), funded by the European Research Council. 

\newpage

% include generated text of all theories
\input{session}

\bibliographystyle{abbrv}
\bibliography{root}

\end{document}
