\documentclass[11pt,a4paper]{article}
\usepackage[T1]{fontenc}
\usepackage{isabelle,isabellesym}
\usepackage{amssymb}

% this should be the last package used
\usepackage{pdfsetup}

% urls in roman style, theory text in math-similar italics
\urlstyle{rm}
\isabellestyle{it}


\begin{document}

\title{Euler's Polynedron Formula}
\author{Lawrence C. Paulson}
\maketitle

\begin{abstract}
Euler stated in 1752 that every convex polyhedron satisfied the formula $V - E + F = 2$ 
where $V$, $E$ and $F$ are the numbers of its vertices, edges, and faces. For three dimensions,
the well-known proof involves removing one face and then flattening the remainder to form a planar graph,
which then is iteratively transformed to leave a single triangle. The history of that proof is extensively
discussed and elaborated by Imre Lakatos~\cite{lakatos}, leaving one finally wondering whether
the theorem even holds. The formal proof provided here has been ported from HOL Light, 
where it is credited to Lawrence~\cite{lawrence-short}. The proof generalises Euler's observation
from solid polyhedra to convex polytopes of arbitrary dimension.
\end{abstract}

\newpage
\tableofcontents

\paragraph*{Acknowledgements}
The author was supported by the ERC Advanced Grant ALEXANDRIA (Project 742178) funded by the European Research Council. 

\newpage

% include generated text of all theories
\input{session}

\bibliographystyle{abbrv}
\bibliography{root}

\end{document}
