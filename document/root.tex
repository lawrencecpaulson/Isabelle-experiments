\documentclass[11pt,a4paper]{article}
\usepackage[T1]{fontenc}
\usepackage{isabelle,isabellesym}
\usepackage{amssymb}
\usepackage{stmaryrd}

% this should be the last package used
\usepackage{pdfsetup}

% urls in roman style, theory text in math-similar italics
\urlstyle{rm}
\isabellestyle{it}


\begin{document}

\title{Wetzel's Problem from THE BOOK}
\author{Lawrence C. Paulson}
\maketitle

\begin{abstract}
Let $F$ be a set of analytic functions on the complex plane such that, 
for each $z\in\mathbb{C}$, the set $\{f(z) \mid f\in F\}$ is countable;
must then $F$ itself be countable?
The answer is yes if the Continuum Hypothesis is false, i.e.,
if the cardinality of $\mathbb{R}$ exceeds $\aleph_1$.
But if CH is true then such an $F$, of cardinality $\aleph_1$,
can be constructed by transfinite recursion.

The formal proof illustrates reasoning about complex 
analysis (analytic and homomorphic functions) and set theory
(transfinite cardinalities) in a single setting.
The mathematical text comes from \emph{Proofs from THE BOOK}~\cite[pp.\thinspace137--8]{aigner-proofs}, by Aigner and Ziegler. 
\end{abstract}

\newpage
\tableofcontents

\paragraph*{Acknowledgements}
The author was supported by the ERC Advanced Grant ALEXANDRIA (Project 742178) funded by the European Research Council. 
Thanks also to Manuel Eberl for advice on proving a function to be holomorphic.

\newpage

% include generated text of all theories
\input{session}

\bibliographystyle{abbrv}
\bibliography{root}

\end{document}
