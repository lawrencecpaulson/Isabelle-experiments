% This is samplepaper.tex, a sample chapter demonstrating the
% LLNCS Version 2.21 of 2022/01/12
%
\documentclass[runningheads]{llncs}
%
\usepackage[T1]{fontenc}
\usepackage{amsmath,amssymb}
%
\usepackage{graphicx}
% Used for displaying a sample figure. If possible, figure files should
% be included in EPS format.

% If you use the hyperref package, please uncomment the following two lines
% to display URLs in blue roman font according to Springer's eBook style:
\usepackage{hyperref,color}
\renewcommand\UrlFont{\color{blue}\rmfamily}

\begin{document}
%
\title{Wetzel: Formalisation of an Undecidable Problem}
%
%\titlerunning{Abbreviated paper title}
% If the paper title is too long for the running head, you can set
% an abbreviated paper title here
%
\author{Lawrence C Paulson\orcidID{0000-0003-0288-4279}}
\institute{Computer Laboratory. University of Cambridge, UK\\
\email{lp15@cam.ac.uk} \quad
\url{https://www.cl.cam.ac.uk/~lp15/}}
%
\authorrunning{L C Paulson}
% First names are abbreviated in the running head.
% If there are more than two authors, 'et al.' is used.
%
%
\maketitle              % typeset the header of the contribution
%
\begin{abstract}
In 1964, Paul Erd\H{o}s published a paper~\cite{erdos-interpolation} settling a question that he had seen in a problem book: given a family $\{f_\alpha\}$ of distinct analytic functions on $\mathbb{C}$ such that for each $z$ the set  $\{f_\alpha ( z )\}$ is countable, is the family itself countable~\cite{garcia-wetzels-problem}?
Erd\H{o}s proved that the answer to the question above was yes if and only if the continuum hypothesis failed.
Therefore, the answer to this innocent-looking question turns out to be undecidable from the axioms of ZFC\@.
The formalisation of these proofs in Isabelle/HOL demonstrate the combination of complex analysis (analytic functions) and set theory (a transfinite construction).
%% TODO Uncomment keywords (commented to allow LIVE typesetting)
%\keywords{Formalisation of mathematics\and  continuum hypothesis \and Isabelle/HOL}
\end{abstract}
%
%
%
\section{First Section}
\subsection{A Subsection Sample}




he continuum hypothesis (CH)


\subsubsection{Sample Heading (Third Level)} Only two levels of
headings should be numbered. Lower level headings remain unnumbered;
they are formatted as run-in headings.

\paragraph{Sample Heading (Fourth Level)}
The contribution should contain no more than four levels of
headings. Table~\ref{tab1} gives a summary of all heading levels.



\noindent Displayed equations are centered and set on a separate
line.
\begin{equation}
x + y = z
\end{equation}
Please try to avoid rasterized images for line-art diagrams and
schemas. Whenever possible, use vector graphics instead (see
Fig.~\ref{fig1}).


\begin{theorem}
This is a sample theorem. The run-in heading is set in bold, while
the following text appears in italics. Definitions, lemmas,
propositions, and corollaries are styled the same way.
\end{theorem}
%
% the environments 'definition', 'lemma', 'proposition', 'corollary',
% 'remark', and 'example' are defined in the LLNCS documentclass as well.
%
\begin{proof}
Proofs, examples, and remarks have the initial word in italics,
while the following text appears in normal font.
\end{proof}
For citations of references, we prefer the use of square brackets
and consecutive numbers. Citations using labels or the author/year
convention are also acceptable. The following bibliography provides
a sample reference list with entries for journal
articles~\cite{ref_article1}, an LNCS chapter~\cite{ref_lncs1}, a
book~\cite{ref_book1}, proceedings without editors~\cite{ref_proc1},
and a homepage~\cite{ref_url1}. Multiple citations are grouped
\cite{ref_article1,ref_lncs1,ref_book1},
\cite{ref_article1,ref_book1,ref_proc1,ref_url1}.

\subsubsection{Acknowledgements} 
This work was supported by the ERC Advanced Grant ALEXANDRIA (Project GA 742178). 

\bibliographystyle{splncs04}
\bibliography{string,atp,general,isabelle,theory,crossref}



\end{document}
