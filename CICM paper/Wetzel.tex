\documentclass[runningheads]{llncs}
% LLNCS Version 2.21 of 2022/01/12
%
\usepackage[T1]{fontenc}
\usepackage[utf8]{inputenc}
\usepackage{amsmath,amssymb}
\usepackage{isabelle,isabellesym,stmaryrd}
\isabellestyle{it}
%
\usepackage{graphicx}
% Used for displaying a sample figure. If possible, figure files should
% be included in EPS format.

% If you use the hyperref package, please uncomment the following two lines
% to display URLs in blue roman font according to Springer's eBook style:
\usepackage{hyperref,color}
\renewcommand\UrlFont{\color{blue}\rmfamily}

\newcommand{\V}{\mathbf{V}}
\newcommand{\bool}{\mathrm{bool}}

\hyphenation{Isa-belle man-u-script man-u-scripts ap-pen-dix mut-u-al-ly co-induc-tive co-induc-tion}

\begin{document}
%
\title{Wetzel: Formalisation of an Undecidable Problem}
%
%\titlerunning{Abbreviated paper title}
% If the paper title is too long for the running head, you can set
% an abbreviated paper title here
%
\author{Lawrence C Paulson\orcidID{0000-0003-0288-4279}}
\institute{Computer Laboratory, University of Cambridge, UK\\
\email{lp15@cam.ac.uk} \quad
\url{https://www.cl.cam.ac.uk/~lp15/}}
%
\authorrunning{L C Paulson}
% First names are abbreviated in the running head.
% If there are more than two authors, 'et al.' is used.
%
%
\maketitle              % typeset the header of the contribution
%
\begin{abstract}
In 1964, Paul Erd\H{o}s published a paper~\cite{erdos-interpolation} settling a question about function spaces that he had seen in a problem book.
Erd\H{o}s proved that the answer was yes if and only if the continuum hypothesis failed.
Therefore, an innocent-looking question turned out to be undecidable from the axioms of ZFC\@.
The formalisation of these proofs in Isabelle/HOL demonstrate the combination of complex analysis and set theory, and in particular how the Isabelle/HOL framework for ZFC is integrated with its basis in higher-order logic.
%% TODO Uncomment keywords (commented to allow LIVE typesetting)
%\keywords{Formalisation of mathematics\and  continuum hypothesis \and Isabelle/HOL}
\end{abstract}


\section{Introduction}

This story~\cite{garcia-wetzels-problem} expresses the richness of mathematics.
It seemed that Paul Erd\H{o}s found the following question in a problem book belonging to the mathematics department of Ann Arbor University:
\begin{quote}
Suppose that $F$ is a family of analytic functions on $\mathbb{C}$ such that for each $z$ the set $\{f (z): f\in F\}$ is countable. (Call this \emph{property $P_0$}.) Then is the family $F$ itself countable?
\end{quote}
This question apparently arose in the PhD work of John E. Wetzel, in connection with spaces of harmonic functions.
Erd\H{o}s was able to show that if the continuum hypothesis failed, every family satisfying $P_0$ had to be countable; but if the hypothesis held, he showed how to create an uncountable family satisfying $P_0$ by a transfinite construction.
His proof appears in \textit{Proofs from THE BOOK}~\cite{aigner-proofs},%
\footnote{End of Chapter 19. Also  \url{https://doi.org/10.1007/978-3-662-57265-8_19}}
Aigner and Ziegler's collection of ``perfect proofs'' inspired by  Erd\H{o}s.

Cantor's celebrated \textit{continuum hypothesis} (CH), number one on Hilbert's list of fundamental questions, asks whether there exists a cardinality between that of the natural numbers, namely $\aleph_0$, and that of the real numbers, namely~$\mathfrak{c}$.
Since the next cardinal after $\aleph_0$ is denoted $\aleph_1$ and the cardinality of the continuum is known to equal $2^{\aleph_0}$,  CH can be written symbolically as $\aleph_1 = 2^{\aleph_0}$.
It's fundamental because the notion of a countable set is straightforward, as is the proof by diagonalisation that the real numbers cannot be enumerated (Cantor's theorem).
CH asserts that no set exists of intermediate size between the natural numbers and the reals.
It was shown to be consistent with the axioms of set theory by Gödel and to be independent from them by Cohen.
The details can be found in any set theory text~\cite{kunen80}.

I have formalised Gödel's model for the former result, the constructible sets~\cite{paulson-consistency}, while
Gunther et al.\ \cite{Independence_CH-AFP} have formalised model constructions both to confirm and refute CH using forcing.
These formalisation tasks were done using Isabelle/ZF, the instance of Isabelle for first-order logic and set theory.
However, Isabelle/HOL is much better developed than Isabelle/ZF;
in particular, the Wetzel problem requires its complex analysis library. So this paper is demonstrates how to tackle a problem that combines the worlds of analysis and set theory, including such mysteries as holomorphic functions, transfinite cardinals and recursion up to uncountable ordinals, on the common basis of higher-order logic.

% TODO paper outline

\section{Isabelle and Set Theory}

Isabelle is a generic theorem prover, ultimately based on a uniform representation of logical syntax in the typed lambda calculus, with inference rules expressed syntactically and combined using higher-order unification~\cite{paulson-found}. Although Isabelle/HOL~\cite{isa-tutorial}---the version for higher-order logic---is by far the best known and most developed instance of Isabelle, other instances exist. These include Isabelle/FOL (classical first-order logic) and Isabelle/ZF\@.
The latter is a faithful development of set theory from the Zermelo-Frawnkel axioms within first-order logic.
Since the axiom of choice (AC) is kept separate from the other axioms, one can investigate weaker forms of AC and equivalents of AC~\cite{paulson-gr}. 

During the 1990s, I made some significant formal developments using Isabelle/ZF~\cite{paulson-reflection}, culminating in a proof of the relative consistency of the axiom of choice using Gödel's constructible universe~\cite{paulson-consistency}. Recently, several highly impressive formalisations of forcing were done within Isabelle/ZF~\cite{gunther-forcing,Independence_CH-AFP}.

On the other hand, Isabelle/ZF lacks much of the automation found in Isabelle/HOL and has no theory even of the real numbers.
That makes it unsuitable for the Wetzel problem. However, it is also possible to formalise ZF set theory within higher-order logic.

\subsection{Formalising ZF in Higher-Order Logic}

% TODO should types be italic?
During the 1990s, Michael JC Gordon conducted several experiments~\cite{gordon-set-theory} involving the formalisation of ZF set theory in his HOL proof assistant. 
His HOL-ST simply introduced a type $\V$ of all sets and a relation ${\in}:\V\times \V\to \bool$, then asserted all the Zermelo-Fraenkel axioms. 
Sten Agerholm~\cite{agerholm-comparison} formalised Dana Scott's inverse limit construction of the set $D_\infty$, satisfying
$D_\infty \cong [D_\infty\to D_\infty]$ and yielding a model of the untyped $\lambda$-calculus.

The use of higher-order logic as opposed to first-order logic as the basis makes this version of set theory somewhat stronger than standard ZF\@. 
In HOL-ST it is possible to define the syntax of first-order logic and the semantics of set-theoretic formulas in terms of~$\V$, verifying the ZF axioms and thus proving their consistency.
On the other hand, a model for HOL-ST can be constructed in ZF plus one inaccessible cardinal. 
These remarks (which Gordon credited to Kenneth Kunen) together imply that 
the strength of HOL-ST is somewhere between ZF and ZF plus one inaccessible cardinal.
This is a weak assumption, much weaker than the dependent type theories used in Coq and Lean, which are stronger than any finite number of inaccessible cardinals~\cite{werner-sets-types}.

Some time later, Steven Obua performed a similar experiment~\cite{obua-partizan-games}, using Isabelle/HOL\@.
He adopted the same axioms and overall approach as Gordon, and demonstrated his system by formalising John H Conway's \textit{partizan games}~\cite{schleicher-conways-games}.
Obua obtained some interoperability with the existing infrastructure of Isabelle/HOL, such as its recursive function definitions. 

\subsection{Axiomatic Type Classes}

Deeper integration requires the use of axiomatic type classes, introduced by Wenzel in 1997.
An \textit{axiomatic type class}~\cite{wenzel-type} defines an open-ended collection of types on the basis of a signature and a possibly empty list of axioms. The signature specifies certain operations and their types, which will be polymorphic with type variables referring to that type class.
The operations may also be equipped with concrete syntax, such as infix declarations.
Any type, whether already existing or defined in the future, can be shown to be an instance of the class if it provides definitions of all the operations in the signature that satisfy the associated axioms.

Isabelle/HOL defines a series of type classes for orderings:
\begin{itemize}
	\item \isa{ord} simply introduces the relations $\le$ and $<$, of type \isa{'a\isasymRightarrow'a\isasymRightarrow bool}
	\item \isa{preorder} extends \isa{ord} with axioms for reflexivity and transitivity
	\item \isa{order} extends \isa{preorder} with antisymmetry
	\item \isa{linorder} extends \isa{order} with the axiom \isa{x\isasymle y \isasymor\ y\isasymle x}
\end{itemize}
Type \isa{nat} (the natural numbers) belongs to each of these classes, and the type operator $\times$ preserves membership in each of these classes.

\subsection{The development ZFC-in-HOL}

My own framework~\cite{ZFC_in_HOL-AFP} for set theory within Isabelle/HOL had two aims:
\begin{enumerate}
	\item To reproduce as much of Isabelle/ZF as possible,
	\item while achieving maximum integration with the underlying higher-order logic.
\end{enumerate}
To achieve these aims I relied on axiomatic type classes, whenever possible overloading existing symbols for set theory rather than introducing new vocabulary.

It introduces the type \isa{V} of sets and the function \isa{elts} of type \isa{V~\isasymRightarrow~V~set} mapping  a set to its elements.
Thus it uses Isabelle/HOL's existing typed sets to represent classes, but not all elements of \isa{V set} correspond to sets.
The predicate \isa{small} identifies those that are small enough, but here comes the first stage of integration: \isa{small} needs to be polymorphic, accepting a set of any type and with the more general meaning that a set is small if its elements can be put into one-to-one correspondence with the elements of a ZF set.
A set that is small in this more general sense does not in itself denote a ZF set, but this condition is frequently necessary if it is used in constructions that ultimately lead to a ZF set.

The fundamentals of set theory are built up as usual, starting with the ZF axioms.
Set membership is expressed using the existing (typed) set membership operator: \isa{x~\isasymin~elts y}.
Union, intersection and the subset relation are expressed using the type class \isa{distrib\_lattice} (distributive lattices), which already provides the symbols \isa{\isasymsqunion}, \isa{\isasymsqinter}, \isa{\isasymle}, etc.
Unions and intersections of families rely on the type class \isa{conditionally\_complete\_lattice}, which provides the symbols \isa{\isasymSqunion} and~\isa{\isasymSqinter}.
Type classes also allow the overloading of \isa{0} to denote the empty set (which is also the natural number zero) and \isa{1} for the natural number one.

On this foundation, it was easy to import significant chunks of Isabelle/ZF: above all, cardinal arithmetic and  the $\aleph$ operator. It was also possible to reuse Isabelle/HOL's existing theory of recursion to obtain transfinite recursion on ordinals, and hence order types, Cantor normal form and much else. So the first aim was met, but more needed to be done to achieve the second.

\subsection{The integration of ZFC-in-HOL with Isabelle/HOL}

In theory, ZFC suffices for the formalisation of all the mathematics in this problem, and much more. In practice, we absolutely do not want to be forced to develop complex analysis from first principles in set theory when it already exists in the Isabelle/HOL libraries. 
And while transfinite cardinalities and other constructions are typically understood from the framework of ZFC, they are perfectly intelligible in a broader context.

The simplest generalisation is for cardinality. The \textit{cardinality} $|x|$ of a ZF set~$x$ is simply the minimal ordinal that is equipollent to\footnote{in one-to-one correspondence with}~$x$. This definition generalises naturally to sets of any type. We can now refer to the cardinality of sets of real and complex numbers.

Next comes \textit{transfinite recursion}, also known as $\epsilon$-recursion. It is a sort of fixedpoint operator allowing the definition of functions over the whole of V\@. If a function $H$ is given, then transfinite recursion yields $F$ such that for any $a$,
\begin{align}
F(a) = H (F\restriction a), \label{eqn:transrec}
\end{align}
where $F\restriction a$ denotes $F$ itself, restricted to the elements of~$a$. It is just an instance of well-founded recursion on the membership relation. The definition is simple:
\begin{isabelle}
\isacommand{definition}\ transrec\ ::\ "((V\ \isasymRightarrow \ 'a)\ \isasymRightarrow \ V\ \isasymRightarrow \ 'a)\ \isasymRightarrow \ V\ \isasymRightarrow \ 'a"\isanewline
\ \ \isakeyword{where}\ "transrec\ H\ a\ \isasymequiv \ wfrec\ \{(x,y).\ x\ \isasymin \ elts\ y\}\ H\ a"
\end{isabelle}
This version differs from the original~\cite{ZFC_in_HOL-AFP} only in its type, which is now polymorphic as shown, allowing the recursively defined function to return anything.
Here, \isa{wfrec} is Isabelle/HOL's built-in operator for well-founded recursion.
The recursion equation~(\ref{eqn:transrec}) easily follows.
\begin{isabelle}
\isacommand{lemma}\ transrec:\ "transrec\ H\ a\ =\ H\ (\isasymlambda x\ \isasymin \ elts\ a.\ transrec\ H\ x)\ a"
\end{isabelle}
Although transfinite recursion is typically used to define operations within the set-theoretic universe~$\V$, we can now use it to create an uncountable set of analytic functions.

\subsection{Embedding Isabelle/HOL Types into $\V$}

The ZFC-in-HOL library defined the class of types that can be \textit{embedded} into the set theoretic universe, \isa{V}, by some injective map, \isa{V\_of}:
\begin{isabelle}
\isacommand{class}\ embeddable\ =\isanewline
\ \ \isakeyword{assumes}\ ex\_inj:\ "\isasymexists V\_of\ ::\ 'a\ \isasymRightarrow \ V.\ inj\ V\_of"
\end{isabelle}
If a type is embeddable then each of its elements corresponds to some ZF set.

As it happens, Isabelle/HOL already provides the class \isa{countable} of all types that can be embedded into the natural numbers. The latter are trivially embedded into \isa{V} as finite ordinals, so it is easy to show that all countable types are \isa{embeddable}.
Examples include \isa{nat}, \isa{int}, \isa{rat}, \isa{bool}.
Trivially, \isa{V} can be embedded into itself, and the type constructors $\times$, $+$ and \isa{list} are straightforwardly shown to preserve the embeddable property.

The library also defines the class of types that are small, which means that the type itself corresponds to some ZF set. It is defined in terms of the predicate \isa{small}:
\begin{isabelle}
\isacommand{class}\ small\ =\isanewline
\ \ \isakeyword{assumes}\ small:\ "small\ (UNIV::'a\ set)"
\end{isabelle}

Every \isa{countable} type (and in particular those listed above) is \isa{small}, because $\omega$ is a set. And it's obvious that every \isa{small} type is \isa{embeddable}. By proving types to be small, we further extend the embeddable class.
The type constructors $\times$, $+$ and \isa{list} preserve smallness.
The situation for the function type constructor ($\to$) is a little subtle:
\begin{isabelle}
\isacommand{instance}\ "fun"\ ::\ (small,embeddable)\ embeddable\isanewline
\isacommand{instance}\ "fun"\ ::\ (small,small)\ small\end{isabelle}
The (straightforward) proofs are omitted.
It should be obvious that we cannot expect to embed $\V\to\V$ into $\V$, but any function $f\in A\to\V$ must be a set provided $A$ is a set.

Types \isa{real} and \isa{complex} are also \isa{small}. How do we know? The reals are obtained by quotienting type \isa{nat \isasymRightarrow\ rat}. The details of their construction do not concern us.
This is the full text of the proof.
\begin{isabelle}
\isacommand{instance}\ real\ ::\ small\ \isanewline
\isacommand{proof}\ -\isanewline
\ \ \isacommand{have}\ "small\ (range\ (Rep\_real))"\isanewline
\ \ \ \ \isacommand{by}\ simp\isanewline
\ \ \isacommand{then}\ \isacommand{show}\ "OFCLASS(real,\ small\_class)"\isanewline
\ \ \ \ \isacommand{by}\ intro\_classes\isanewline
\ \ \ \ \ \ \ (metis\ Rep\_real\_inverse\ image\_inv\_f\_f\ inj\_on\_def\ replacement)\isanewline
\isacommand{qed}
\end{isabelle}
And type \isa{complex} is essentially the same thing as $\mathbb{R}^2$.

\section{Wetzel's Problem: the $\neg$CH Case}

Now it is time to formalise the proof itself, following Aigner and Ziegler's presentation~\cite{aigner-proofs}.
We begin by defining the predicate \isa{Wetzel}, corresponding to $P_0$ above, on sets of complex-valued functions. It holds if every element of the given set $F$ is analytic on the complex plane and if, for all $z$, the set $\{f(z) : f\in F\}$ is countable:
\begin{isabelle}
\isacommand{definition}\ Wetzel\ ::\ "(complex\ \isasymRightarrow \ complex)\ set\ \isasymRightarrow \ bool"\isanewline
\ \ \isakeyword{where}\ "Wetzel\ \isasymequiv \ \isasymlambda F.\ (\isasymforall f\isasymin F.\ f\ analytic\_on\ UNIV)\ \isasymand\isanewline
\ \ \ \ \ \ \ \ \ \ \ \ \ \ \ \ \ \ \ \ \ \ \ \ (\isasymforall z.\ countable((\isasymlambda f.\ f\ z)\ `\ F))"
\end{isabelle}

Remarkably, the Isabelle/HOL proof is barely 50 lines. First, the statement:
\begin{isabelle}
\isacommand{proposition}\ Erdos\_Wetzel\_nonCH:\isanewline
\ \ \isakeyword{assumes}\ W:\ "Wetzel\ F"\ \isakeyword{and}\ NCH:\ "C\_continuum\ >\ \isasymaleph 1"\ \isakeyword{and}\ "small\ F"\isanewline
\ \ \isakeyword{shows}\ "countable\ F"
\end{isabelle}

The key step is to prove the contrapositive of the claim:
\begin{isabelle}
\ \ \isacommand{have}\ "\isasymexists z0.\ gcard\ ((\isasymlambda f.\ f\ z0)\ `\ F)\ \isasymge \ \isasymaleph 1"\ \isakeyword{if}\ "uncountable\ F"
\end{isabelle}

\noindent
Given the uncountable family $F$, we find a subset $F'\subseteq F$ of cardinality~$\aleph_1$  and thus a bijection $\phi:\aleph_1\to F'$  between the ordinals below $\aleph_1$ and $F'$.
\begin{isabelle}
\ \ \ \ \isacommand{have}\ "gcard\ F\ \isasymge \ \isasymaleph 1"\isanewline
\ \ \ \ \ \ \isacommand{using}\ \isacartoucheopen small\ F\isacartoucheclose \ that\ uncountable\_gcard\_ge\ \isacommand{by}\ blast\ \isanewline
\ \ \ \ \isacommand{then}\ \isacommand{obtain}\ F'\ \isakeyword{where}\ "F'\ \isasymsubseteq \ F"\ \isakeyword{and}\ F':\ "gcard\ F'\ =\ \isasymaleph 1"\isanewline
\ \ \ \ \ \ \isacommand{by}\ (meson\ Card\_Aleph\ Ord\_1\ subset\_smaller\_gcard\ \isacartoucheopen small\ F\isacartoucheclose )\isanewline
\ \ \ \ \isacommand{then}\ \isacommand{obtain}\ \isasymphi \ \isakeyword{where}\ \isasymphi :\ "bij\_betw\ \isasymphi \ (elts\ \isasymomega 1)\ F'"\isanewline
\ \ \ \ \ \ \isacommand{by}\ (metis\ TC\_small\ eqpoll\_def\ gcard\_eqpoll)
\end{isabelle}

\noindent
We next define the family of sets $S(\alpha,\beta)$ as  $\{z.\, \phi_\alpha (z) = \phi_\beta (z)\}$.
Here $\alpha$ and $\beta$ range over ordinals, and $\omega_1$ is the first uncountable ordinal (thus the same thing as $\aleph_1$, but regarded as an ordinal).
The $\phi_\alpha$ for $\alpha<\omega_1$ are the given analytic functions.
\begin{isabelle}
\ \ \ \ \isacommand{define}\ S\ \isakeyword{where}\ "S\ \isasymequiv \ \isasymlambda \isasymalpha \ \isasymbeta .\ \{z.\ \isasymphi \ \isasymalpha \ z\ =\ \isasymphi \ \isasymbeta \ z\}"
\end{isabelle}

It takes 10 lines to prove that $S(\alpha,\beta)$ is countable for $\alpha<\beta<\omega_1$,
since two distinct holomorphic functions can agree at only countably many arguments.
\begin{isabelle}
\ \ \ \ \isacommand{have}\ "gcard\ (S\ \isasymalpha \ \isasymbeta )\ \isasymle \ \isasymaleph 0"\ \isakeyword{if}\ "\isasymalpha\ \isasymin\ elts\ \isasymbeta "\ "\isasymbeta\ \isasymin\ elts \isasymomega 1"\ \isakeyword{for}\ \isasymalpha \ \isasymbeta
\end{isabelle}

% TODO fix union syntax
The next step is to define $SS$ as the union of all $S(\alpha,\beta)$ for $\alpha<\beta<\omega_1$.
\begin{isabelle}
\ \ \ \ \isacommand{define}\ SS\ \isakeyword{where}\ "SS\ \isasymequiv \ \isasymSqunion ((\isasymlambda \isasymbeta .\ \isasymSqunion ((\isasymlambda \isasymalpha .\ S\ \isasymalpha \ \isasymbeta )\ `\ elts\ \isasymbeta ))\ `\ elts \isasymomega 1)"
\end{isabelle}

A 14 line calculation shows that $|SS|\le\aleph_1$, but we are assuming the negation of CH, so $SS$ cannot be the entire complex plane: there exists some $z_0\not\in SS$.
\begin{isabelle}
\ \ \ \ \isacommand{finally}\ \isacommand{have}\ "gcard\ SS\ \isasymle \ \isasymaleph 1"\ \isacommand{.}\isanewline
\ \ \ \ \isacommand{with}\ NCH\ \isacommand{obtain}\ z0\ \isakeyword{where}\ "z0\ \isasymnotin \ SS"\isanewline
\ \ \ \ \ \ \isacommand{by}\ (metis\ Complex\_gcard\ UNIV\_eq\_I\ less\_le\_not\_le)
\end{isabelle}

That $z_0$ satisfies our requirements follows straightforwardly by the definitions of $S$ and $SS$.

\section{Wetzel's Problem: the CH Case}

Assuming CH, it is possible to construct an uncountable family of analytic functions that makes the Wetzel property $P_0$ fail.
It's a rather delicate transfinite recursion, so let's review the argument before examining the formal proof.

\subsection{The transfinite construction}

CH implies that $|\mathbb{C}| = \aleph_1$ and we can write $\mathbb{C} = \{\zeta_\alpha : \alpha < \omega_1 \}$.
Now consider the set $D\subseteq\mathbb{C}$ of \textit{rational} complex numbers: 
\begin{align*}
 D = \{p+iq: p,q\in\mathbb{Q}\}.
\end{align*}
Suppose that we had a family of functions 
 $\{f_\beta : \beta < \omega_1 \}$ such that 
\begin{align}\label{eqn:f_in_D}
	f_\beta (\zeta_\alpha) \in D\quad\text{if}\quad\alpha<\beta.
\end{align} 
 %
 Since the set $D$ is countable and the $\zeta_\alpha$ for $\alpha < \omega_1$ include all the complex numbers, the desired result would follow.
Erd\H{o}s showed how to construct this family by transfinite induction: for an arbitrary $\gamma<\omega_1$, assume that a family of distinct analytic functions $\{f_\beta: \beta<\gamma\}$ defined below~$\gamma$.
To complete the construction and therefore prove the theorem, we must extend it with a new function, $f_\gamma$.

Since $\gamma$ is countable, the set $\{f_\beta: \beta<\gamma\}$ can be enumerated as $\{g_0, g_1, \ldots\}$
and $\{\zeta_\alpha : \alpha < \gamma\}$ can be enumerated as $\{w_0, w_1, \ldots\}$;
both are finite or infinite according to whether $\gamma$ itself is finite or infinite.
The sought-for analytic function $f_\gamma$ should satisfy, for all $n$,
\begin{align}
	f_\gamma (w_n) \in D \quad\text{and}\quad f_\gamma (w_n) \not= g_n(w_n). \label{eqn:f_induct}
\end{align}
%
The second condition above ensures that $f_\gamma$ is new, while the first is simply~(\ref{eqn:f_in_D}).
We construct $f_\gamma$ by putting
\begin{align}
\begin{split}
f_\gamma(z) & := \epsilon_0 + \epsilon_1 (z - w_0) + \epsilon_2 (z - w_0)(z - w_1)\\
	       & \quad + \epsilon_3 (z - w_0)(z - w_1)(z - w_2) + \cdots.	\label{eqn:f_gamma}
\end{split}
\end{align}
Again, this sum is finite or infinite according to $\gamma$. 
The $\{\epsilon_m\}$ are complex numbers chosen one at a time to satisfy conditions~(\ref{eqn:f_induct}) above; 
ensuring that $f_\gamma (w_n)$ avoids $g_n(w_n)$ is possible because $D$ is dense in~$\mathbb{C}$\@.
In the finite case the sum is a polynomial, so trivially analytic. 
In the infinite case, the choice of the $\{\epsilon_m\}$ needs to be carefully calibrated in order to satisfy the conditions while converging to zero sufficiently rapidly. 
We have a lot of leeway and I chose $$|\epsilon_m| < \bigl[m! \cdot \prod_{i<m} (1 + |w_i|)\bigr]^{-1}. $$
The summation~(\ref{eqn:f_gamma}) converges to an analytic function because the uniform limit of holomorphic functions is holomorphic, and the limit can be shown to be uniform by the Weierstrass $M$-test.%
\footnote{Thanks to Manuel Eberl for suggesting this argument.}

A clever aspect of the construction is that the conditions~(\ref{eqn:f_induct}) constrain only $f_\gamma(w_n)$, 
whose value depends only on $\epsilon_m$ for $m\le n$. That is, $f_\gamma(w_0) = \epsilon_0$, $f_\gamma(w_1) = \epsilon_0 + \epsilon_1 (w_1 - w_0)$, etc.
The desired values of $\epsilon_m$ can be calculated sequentially.

\subsection{The Isabelle/HOL Formalisation}

\newcommand{\prf}{$\langle$proof$\,\rangle$}

Now, let's state the theorem formally:
\begin{isabelle}
\isacommand{proposition}\ Erdos\_Wetzel\_CH:\isanewline
\ \ \isakeyword{assumes}\ CH:\ "C\_continuum\ =\ \isasymaleph 1"\isanewline
\ \ \isakeyword{obtains}\ F\ \isakeyword{where}\ "Wetzel\ F"\ \isakeyword{and}\ "uncountable\ F"
\end{isabelle}
%
The proof begins with a self-evident definition of $D$. It's then shown to be countably infinite.
\begin{isabelle}
\ \ \isacommand{define}\ D\ \isakeyword{where}\ "D\ \isasymequiv \ \{z.\ Re\ z\ \isasymin \ \isasymrat \ \isasymand \ Im\ z\ \isasymin \ \isasymrat \}"\isanewline
\ \ \isacommand{have}\ Deq:\ "D\ =\ (\isasymUnion x\isasymin \isasymrat .\ \isasymUnion y\isasymin \isasymrat .\ \{Complex\ x\ y\})"\isanewline
\ \ \ \ \isacommand{using}\ complex.collapse\ \isacommand{by}\ (force\ simp:\ D\_def)\isanewline
\ \ \isacommand{with}\ countable\_rat\ \isacommand{have}\ "countable\ D"\isanewline
\ \ \ \ \isacommand{by}\ blast\isanewline
\ \ \isacommand{have}\ "infinite\ D"\isanewline
\ \ \ \ \prf\isanewline
\ \ \isacommand{have}\ "\isasymexists w.\ Re\ w\ \isasymin \ \isasymrat \ \isasymand \ Im\ w\ \isasymin \ \isasymrat \ \isasymand \ norm\ (w\ -\ z)\ <\ e"\ \isakeyword{if}\ "e\ >\ 0"\ \isakeyword{for}\ z\ e\isanewline
\ \ \ \ \prf\isanewline
\ \ \isacommand{then}\ \isacommand{have}\ cloD:\ "closure\ D\ =\ UNIV"\isanewline
\ \ \ \ \isacommand{by}\ (auto\ simp:\ D\_def\ closure\_approachable\ dist\_complex\_def)
\end{isabelle}
The closure of $D$ equals the universal set \isa{UNIV} of type \isa{complex set}.
We obtain the transfinite enumeration $\{\zeta_\alpha : \alpha < \omega_1\}$ of the complex plane.
 \begin{isabelle}
\ \ \isacommand{obtain}\ \isasymzeta \ \isakeyword{where}\ \isasymzeta :\ "bij\_betw\ \isasymzeta \ (elts\ \isasymomega 1)\ (UNIV::complex\ set)"\isanewline
\ \ \ \ \isacommand{by}\ (metis\ Complex\_gcard\ TC\_small\ assms\ eqpoll\_def\ gcard\_eqpoll)
\end{isabelle}
%
Next come some technical definitions: \isa{inD} for functions whose range for certain arguments lies within $D$ and $\Phi$ to express that $f$ is a family of analytic functions indexed by the ordinals up to~$\beta$. 

\begin{isabelle}
\ \ \isacommand{define}\ inD\ \isakeyword{where}\ "inD\ \isasymequiv \ \isasymlambda \isasymbeta \ f.\ (\isasymforall \isasymalpha \ \isasymin \ elts\ \isasymbeta .\ f\ (\isasymzeta \ \isasymalpha )\ \isasymin \ D)"\isanewline
\ \ \isacommand{define}\ \isasymPhi \ \isakeyword{where}\ "\isasymPhi \ \isasymequiv\isanewline
\ \  \ \isasymlambda \isasymbeta \ f.\ f\ \isasymbeta \ analytic\_on\ UNIV\ \isasymand \ inD\ \isasymbeta \ (f\ \isasymbeta )\ \isasymand \ inj\_on\ f\ (elts(succ\ \isasymbeta ))"
\end{isabelle}
%
The lemma \isa{*} is the step of the transfinite induction.
A family $f$ defined below the ordinal~$\gamma$ is extended with a new function, $f_\gamma$ (the variable $h$ below).

\begin{isabelle}
\ \ \isacommand{have}\ *:\ "\isasymexists h.\ \isasymPhi \ \isasymgamma \ ((restrict\ f\ (elts\ \isasymgamma ))(\isasymgamma :=h))"\isanewline
\ \ \ \ \isakeyword{if}\ \isasymgamma :\ "\isasymgamma \ \isasymin \ elts\ \isasymomega 1"\ \isakeyword{and}\ "\isasymforall \isasymbeta \ \isasymin \ elts\ \isasymgamma .\ \isasymPhi \ \isasymbeta \ f"\ \isakeyword{for}\ \isasymgamma \ f
\end{isabelle}

\noindent
The construction of $h$ depends on whether or not $\gamma$ is finite:
\begin{isabelle}
\ \ \ \ \isacommand{obtain}\ h\ \isakeyword{where}\ "h\ analytic\_on\ UNIV"\ \ "inD\ \isasymgamma \ h"\ \ "\isasymforall \isasymbeta \ \isasymin \ elts\ \isasymgamma .\ h\ \isasymnoteq \ f\ \isasymbeta"\isanewline
\ \ \ \ \isacommand{proof}\ (cases\ "finite\ (elts\ \isasymgamma )")
\end{isabelle}
%
The finite case is easier, since the function we have to construct will simply be a polynomial, and trivially analytic. The finite ordinal~$\gamma$ is simply some natural number~$n$, and the construction of $f_\gamma$ is by induction on~$n$. 

Let's see some highlights of the infinite case, in which similar ideas are taken to the max.

there's a bijection $\eta$ between $\mathbb{N}$ (an initial segment of $\mathbb{N}$ if $\gamma$ is finite) and $\gamma$ (which equals $\{\alpha:\alpha<\gamma\}$). 


\begin{isabelle}
\ \ \ \ \ \ \isacommand{case}\ False\isanewline
\ \ \ \ \ \ \isacommand{then}\ \isacommand{obtain}\ \isasymeta \ \isakeyword{where}\ \isasymeta :\ "bij\_betw\ \isasymeta \ (UNIV::nat\ set)\ (elts\ \isasymgamma )"\isanewline
\ \ \ \ \ \ \ \ \isacommand{by}\ (meson\ \isasymgamma \ countable\_infiniteE'\ less\_\isasymomega 1\_imp\_countable)\isanewline
\ \ \ \ \ \ \isacommand{define}\ g\ \isakeyword{where}\ "g\ \isasymequiv \ f\ o\ \isasymeta "\isanewline
\ \ \ \ \ \ \isacommand{define}\ w\ \isakeyword{where}\ "w\ \isasymequiv \ \isasymzeta \ o\ \isasymeta "
\end{isabelle}

\begin{isabelle}
\ \ \ \ \ \ \isacommand{define}\ p\ \isakeyword{where}\ "p\ \isasymequiv \ \isasymlambda n\ z.\ \isasymProd i<n.\ z\ -\ w\ i"\isanewline
\ \ \ \ \ \ \isacommand{define}\ q\ \isakeyword{where}\ "q\ \isasymequiv \ \isasymlambda n.\ \isasymProd i<n.\ 1\ +\ norm\ (w\ i)"\isanewline
\ \ \ \ \ \ \isacommand{define}\ h\ \isakeyword{where}\ "h\ \isasymequiv \ \isasymlambda n\ \isasymepsilon \ z.\ \isasymSum i<n.\ \isasymepsilon \ i\ *\ p\ i\ z"
\end{isabelle}

\begin{isabelle}
\ \ \ \ \ \ \isacommand{define}\ BALL\ \isakeyword{where}\ "BALL\ \isasymequiv \ \isasymlambda n\ \isasymepsilon .\ ball\ (h\ n\ \isasymepsilon \ (w\ n))\ (norm\ (p\ n\ (w\ n))\ /\ (fact\ n\ *\ q\ n))"\isanewline
\ \ \ \ \ \ \ \ \ \ \ \ \ \ \ \ \ \ %
\isamarkupcmt{The demonimator above is the key to keeping the epsilons small%
}\isanewline
\ \ \ \ \ \ \isacommand{define}\ DD\ \isakeyword{where}\ "DD\ \isasymequiv \ \isasymlambda n\ \isasymepsilon .\ D\ \isasyminter \ BALL\ n\ \isasymepsilon \ -\ \{g\ n\ (w\ n)\}"\isanewline
\ \ \ \ \ \ \isacommand{define}\ dd\ \isakeyword{where}\ "dd\ \isasymequiv \ \isasymlambda n\ \isasymepsilon .\ SOME\ x.\ x\ \isasymin \ DD\ n\ \isasymepsilon"
\end{isabelle}

\begin{isabelle}
\ \ \ \ \ \ \isacommand{have}\ p0:\ "p\ n\ z\ =\ 0\ \isasymlongleftrightarrow \ (\isasymexists i<n.\ z\ =\ w\ i)"\ \isakeyword{for}\ z\ n\isanewline
\ \ \ \ \ \ \ \ \isacommand{unfolding}\ p\_def\ \isacommand{by}\ force\isanewline
\ \ \ \ \ \ \isacommand{have}\ "DD\ n\ \isasymepsilon \ \isasymnoteq \ \{\}"\ \isakeyword{for}\ n\ \isasymepsilon \isanewline
\ \ \ \ \ \ \ \ \prf\isanewline
\ \ \ \ \ \ \isacommand{then}\ \isacommand{have}\ dd\_in\_DD:\ "dd\ n\ \isasymepsilon \ \isasymin \ DD\ n\ \isasymepsilon "\ \isakeyword{for}\ n\ \isasymepsilon \isanewline
\ \ \ \ \ \ \ \ \isacommand{by}\ (simp\ add:\ dd\_def\ some\_in\_eq)
\end{isabelle}

\begin{isabelle}
\ \ \ \ \ \ \isacommand{define}\ coeff\ \isakeyword{where}\ "coeff\ \isasymequiv\ \ldots"\isanewline
\ \ \ \ \ \ \isacommand{have}\ coeff\_eq:\ "coeff\ n\ =\ (dd\ n\ coeff\ -\ h\ n\ coeff\ (w\ n))\ /\ p\ n\ (w\ n)"\ \isakeyword{for}\ n\isanewline
\ \ \ \ \ \ \ \ \isacommand{by}\ (simp\ add:\ def\_wfrec\ [OF\ coeff\_def])\isanewline
\ \ \ \ \ \ \isacommand{have}\ norm\_coeff:\ "norm\ (coeff\ n)\ <\ 1\ /\ (fact\ n\ *\ q\ n)"\ \isakeyword{for}\ n\isanewline
\ \ \ \ \ \ \ \ \ \prf
\end{isabelle}

\begin{isabelle}
\ \ \ \ \ \ \isacommand{have}\ h\_truncated:\ "h\ n\ coeff\ (w\ k)\ =\ h\ (Suc\ k)\ coeff\ (w\ k)"\ \isakeyword{if}\ "k\ <\ n"\ \isakeyword{for}\ n\ k\isanewline
\ \ \ \ \ \ \ \prf\isanewline
\ \ \ \ \ \ \isacommand{have}\ norm\_p\_bound:\ "norm\ (p\ n\ z')\ \isasymle \ q\ n\ *\ (1\ +\ norm\ z)\ \isacharcircum \ n"\ \isanewline
\ \ \ \ \ \ \ \ \ \ \isakeyword{if}\ "dist\ z\ z'\ \isasymle \ 1"\ \isakeyword{for}\ n\ z\ z'\isanewline
\ \ \ \ \ \ \ \prf
\end{isabelle}

Now we conclude proofs about $f_\gamma$

Weierstrass, also ratio test
\begin{isabelle}
\ \ \ \ \ \ \ \ \isacommand{define}\ hh\ \isakeyword{where}\ "hh\ \isasymequiv \ \isasymlambda z.\ suminf\ (\isasymlambda i.\ coeff\ i\ *\ p\ i\ z)"\isanewline
\ \ \ \ \ \ \ \ \isacommand{have}\ "hh\ holomorphic\_on\ UNIV"\isanewline
\ \ \ \ \ \ \ \ \ \ \prf\isanewline
\ \ \ \ \ \ \ \ \isacommand{then}\ \isacommand{show}\ "hh\ analytic\_on\ UNIV"\isanewline
\ \ \ \ \ \ \ \ \ \ \isacommand{by}\ (simp\ add:\ analytic\_on\_open)
\end{isabelle}


\begin{isabelle}
\ \ \ \ \ \ \ \ \isacommand{have}\ hh\_eq\_dd:\ "hh\ (w\ n)\ =\ dd\ n\ coeff"\ \isakeyword{for}\ n\isanewline
\ \ \ \ \ \ \ \ \ \ \prf\isanewline
\ \ \ \ \ \ \ \ \isacommand{then}\ \isacommand{have}\ "hh\ (w\ n)\ \isasymin \ D"\ \isakeyword{for}\ n\isanewline
\ \ \ \ \ \ \ \ \ \ \isacommand{using}\ DD\_def\ dd\_in\_DD\ \isacommand{by}\ fastforce\isanewline
\ \ \ \ \ \ \ \ \isacommand{then}\ \isacommand{show}\ "inD\ \isasymgamma \ hh"\isanewline
\ \ \ \ \ \ \ \ \ \ \isacommand{unfolding}\ inD\_def\ \isacommand{by}\ (metis\ \isasymeta \ bij\_betw\_iff\_bijections\ comp\_apply\ w\_def)\isanewline
\ \ \ \ \ \ \ \ \isacommand{have}\ "hh\ (w\ n)\ \isasymnoteq \ f\ (\isasymeta \ n)\ (w\ n)"\ \isakeyword{for}\ n\isanewline
\ \ \ \ \ \ \ \ \ \ \isacommand{using}\ DD\_def\ dd\_in\_DD\ g\_def\ hh\_eq\_dd\ \isacommand{by}\ auto\isanewline
\ \ \ \ \ \ \ \ \isacommand{then}\ \isacommand{show}\ "\isasymforall \isasymbeta \isasymin elts\ \isasymgamma .\ hh\ \isasymnoteq \ f\ \isasymbeta "\isanewline
\ \ \ \ \ \ \ \ \ \ \isacommand{by}\ (metis\ \isasymeta \ bij\_betw\_imp\_surj\_on\ imageE)
\end{isabelle}


\begin{isabelle}
X
\end{isabelle}

\begin{isabelle}
X
\end{isabelle}


DOING THE ACTUAL INDUCTION: skipped

FINALLY the puttng all the bits together

\begin{isabelle}
\ \ \isacommand{show}\ ?thesis\isanewline
\ \ \isacommand{proof}\isanewline
\ \ \ \ \isacommand{let}\ ?F\ =\ "f\ `\ elts\ \isasymomega 1"\isanewline
\ \ \ \ \isacommand{have}\ "countable\ ((\isasymlambda f.\ f\ z)\ `\ f\ `\ elts\ \isasymomega 1)"\ \isakeyword{for}\ z\isanewline
\ \ \ \ \isacommand{proof}\ -\isanewline
\ \ \ \ \ \ \isacommand{obtain}\ \isasymalpha \ \isakeyword{where}\ \isasymalpha :\ "\isasymzeta \ \isasymalpha \ =\ z"\ "\isasymalpha \ \isasymin \ elts\ \isasymomega 1"\ "Ord\ \isasymalpha "\isanewline
\ \ \ \ \ \ \ \ \isacommand{by}\ (meson\ Ord\_\isasymomega 1\ Ord\_in\_Ord\ UNIV\_I\ \isasymzeta \ bij\_betw\_iff\_bijections)\isanewline
\ \ \ \ \ \ \isacommand{let}\ ?B\ =\ "elts\ \isasymomega 1\ -\ elts\ (succ\ \isasymalpha )"\isanewline
\ \ \ \ \ \ \isacommand{have}\ eq:\ "elts\ \isasymomega 1\ =\ elts\ (succ\ \isasymalpha )\ \isasymunion \ ?B"\isanewline
\ \ \ \ \ \ \ \ \isacommand{using}\ \isasymalpha \ \isacommand{by}\ (metis\ Diff\_partition\ Ord\_\isasymomega 1\ OrdmemD\ less\_eq\_V\_def\ succ\_le\_iff)\isanewline
\ \ \ \ \ \ \isacommand{have}\ "(\isasymlambda f.\ f\ z)\ `\ f\ `\ ?B\ \isasymsubseteq \ D"\isanewline
\ \ \ \ \ \ \ \ \isacommand{using}\ \isasymalpha \ inD\ \isacommand{by}\ clarsimp\ (meson\ Ord\_\isasymomega 1\ Ord\_in\_Ord\ Ord\_linear)\isanewline
\ \ \ \ \ \ \isacommand{then}\ \isacommand{have}\ "countable\ ((\isasymlambda f.\ f\ z)\ `\ f\ `\ ?B)"\isanewline
\ \ \ \ \ \ \ \ \isacommand{by}\ (meson\ \isacartoucheopen countable\ D\isacartoucheclose \ countable\_subset)\isanewline
\ \ \ \ \ \ \isacommand{moreover}\ \isacommand{have}\ "countable\ ((\isasymlambda f.\ f\ z)\ `\ f\ `\ elts\ (succ\ \isasymalpha ))"\isanewline
\ \ \ \ \ \ \ \ \isacommand{by}\ (simp\ add:\ \isasymalpha \ less\_\isasymomega 1\_imp\_countable)\isanewline
\ \ \ \ \ \ \isacommand{ultimately}\ \isacommand{show}\ ?thesis\isanewline
\ \ \ \ \ \ \ \ \isacommand{using}\ eq\ \isacommand{by}\ (metis\ countable\_Un\_iff\ image\_Un)\isanewline
\ \ \ \ \isacommand{qed}\isanewline
\ \ \ \ \isacommand{then}\ \isacommand{show}\ "Wetzel\ ?F"\isanewline
\ \ \ \ \ \ \isacommand{unfolding}\ Wetzel\_def\ \isacommand{by}\ (blast\ intro:\ anf)\isanewline
\ \ \ \ \isacommand{show}\ "uncountable\ ?F"\isanewline
\ \ \ \ \ \ \isacommand{using}\ Ord\_\isasymomega 1\ countable\_iff\_less\_\isasymomega 1\ countable\_image\_inj\_eq\ injf\ \isacommand{by}\ blast\isanewline
\ \ \isacommand{qed}\isanewline
\isacommand{qed}
\end{isabelle}

\begin{isabelle}
X
\end{isabelle}

\begin{isabelle}
\isacommand{theorem}\ Erdos\_Wetzel:\ "C\_continuum\ =\ \isasymaleph 1\ \isasymlongleftrightarrow \ (\isasymexists F.\ Wetzel\ F\ \isasymand \ uncountable\ F)"\isanewline
\ \ \isacommand{by}\ (metis\ C\_continuum\_ge\ Erdos\_Wetzel\_CH\ Erdos\_Wetzel\_nonCH\isanewline
\ \ \ \ \ \ \ \ \ \ \ \ \ \ \ TC\_small\ less\_V\_def)
\end{isabelle}

\begin{isabelle}
X
\end{isabelle}

\begin{isabelle}
X
\end{isabelle}

\begin{isabelle}
X
\end{isabelle}

\begin{isabelle}
X
\end{isabelle}



% TODO
15 pages including references


\section{Discussion}

\section{Conclusions}

\subsubsection{Acknowledgements} 
This work was supported by the ERC Advanced Grant ALEXANDRIA (Project GA 742178). 
Dmitriy Traytel provided a particularly slick formal axiomatisation of type $V$ and Manuel Eberl provided a crucial tip for the final part.

\bibliographystyle{splncs04}
\bibliography{string,atp,general,isabelle,theory,crossref}



\end{document}
